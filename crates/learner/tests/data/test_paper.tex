\documentclass[conference]{IEEEtran}
\usepackage{amsmath,graphicx,lipsum}

% Enhanced metadata configuration
\usepackage[pdftex,
    pdftitle={Analysis of PDF Extraction Methods},
    pdfauthor={Alice Researcher and Bob Scholar},
    pdfsubject={PDF Content Analysis},
    pdfkeywords={PDF analysis, text extraction, metadata, academic papers},
    pdfproducer={pdfTeX},
    pdfcreator={LaTeX with hyperref}
]{hyperref}

\title{Analysis of PDF Extraction Methods\\
\large{With Multiple Features for Testing}}

\author{
    \IEEEauthorblockN{Alice Researcher\textsuperscript{1}, Bob Scholar\textsuperscript{2}}
    \IEEEauthorblockA{\textsuperscript{1}Department of Computer Science, Example University\\
    Email: alice@example.edu}
    \IEEEauthorblockA{\textsuperscript{2}Institute of Technology, Another University\\
    Email: bob@example.edu}
}

\begin{document}

\maketitle

\begin{abstract}
This is a sample paper designed for testing PDF analysis features. It includes various elements like \textbf{bold text}, \textit{italics}, citations \cite{smith2023}, equations, tables, and figures. The paper has been carefully constructed to test extraction of metadata, structure, and content.
\end{abstract}

\section{Introduction}
\lipsum[1]

Here's a link to \href{https://example.com}{Example Website} and a footnote\footnote{This is a sample footnote.}.

\section{Methods}
\subsection{Experimental Setup}
This section includes an equation:
\begin{equation}
    E = mc^2 \label{eq:einstein}
\end{equation}

And a table:
\begin{table}[ht]
    \caption{Sample Results}
    \centering
    \begin{tabular}{|c|c|c|}
        \hline
        Method & Accuracy & Time (ms) \\
        \hline
        A & 95\% & 100 \\
        B & 92\% & 80 \\
        \hline
    \end{tabular}
    \label{tab:results}
\end{table}

\subsection{Implementation}
\lipsum[2]

\section{Results}
Here's a simple figure:
\begin{figure}[ht]
    \centering
    \rule{3cm}{2cm} % This creates a simple black rectangle
    \caption{A Sample Figure}
    \label{fig:sample}
\end{figure}

As shown in Figure \ref{fig:sample}, we can see that...

\section{Discussion}
Here's a list of key points:
\begin{itemize}
    \item Point A with citation \cite{jones2022}
    \item Point B with equation reference (\ref{eq:einstein})
    \item Point C with figure reference (Fig. \ref{fig:sample})
\end{itemize}

\section{Conclusion}
This test paper demonstrates various PDF features including metadata, structure, cross-references, and different content types.

% Bibliography
\begin{thebibliography}{2}
\bibitem{smith2023} J. Smith, ``Sample Paper Title,'' Journal of Examples, vol. 1, no. 1, pp. 1-10, 2023. DOI: 10.1234/example.2023.001

\bibitem{jones2022} R. Jones, ``Another Sample Paper,'' Proc. Int. Conf. Examples, pp. 100-110, 2022. DOI: 10.5678/conf.2022.123
\end{thebibliography}

\end{document}